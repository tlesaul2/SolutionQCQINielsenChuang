%!TeX root=../solnQCQI.tex

\chapter{Fundamental Concepts}
\phantomsection
\cftaddnumtitleline{toc}{chapter}{}{\normalfont The exercise in this chapter is only interesting for it's mathematics, so it was moved to the end to avoid dissuading non-mathematicians from continuing to chapters more interesting for their quantum information theory.}{}
\Textbf{1.1} Probabilistic Classical Deutsch-Jozsa Algorithm: Suppose that the problem is not to distinguish between the constant and balanced functions \textit{with certainty}, but rather, with some probability of error $\epsilon < 1/2$.  What is the performance of the best classical algorithm for this problem?
\Soln  To a mathematician, this problem is (\textit{slightly}) under-specified.  Missing is the probability that the function $f$ in question is balanced, vice constant.  We assume that both are \textbf{equally} likely, a priori.  The results when all balanced or constant functions are chosen from randomly are significantly different, and likely less interesting.
We describe \textit{an} algorithm and analyze the error rate, but make no effort to show that it is the \textit{best} algorithm, nor that this is the most effective analysis. Let $C$ be the event that $f$ is constant, and $B$ be the event that it is balanced.  By hypothesis $P(C)=P(B)=\frac12$, a  priori.  Evaluating $f$ provides information which can be used to update these prior probabilities.  Classically evaluating the function once, say at $x_0$, provides no useful information, since comparison of values is at the heart of this problem.  Evaluating $f$ twice, say at $x_0$ and $x_1$, can unambiguously determine if $f$ is balanced when their values disagree.  So, let's assume they agree.  We use Bayesian inference to iteratively update the probability that $f$ is constant, given $k$ successive measurements that agree.  In a convenient abuse of notation, let $P(E\ |\ k) = P\bigl(E\ |\ f(x_0) = \cdots = f(x_{k-1})\bigr)$, $P(k\ |\ E) = P\bigl(f(x_0) = \cdots = f(x_{k-1})\ |\ E\bigr)$, and $P(k) =  P\bigl(f(x_0) = \cdots = f(x_{k-1})\bigr)$, for $E=B,C$, and $k\in\mathbb{N}$.  We have $P(C\ |\ 0)=P(C\ |\ 1)=P(B\ |\ 0)=P(B\ |\ 1) = 1/2$.  Note also that $P(k\ |\ C) = 1$, since if $f$ is constant all evaluations (including the $k$ in question) will agree.
By Baye's theorem and the Law of Total Probability:
\begin{align*}
P(C\ |\  k)&=\ \ \ \ \ \ \ \ \ \ \ \ \ \ \ \ \ \ \frac{P(k\ |\  C)\cdot P(C\ |\ k-1)}{P(k)} \\
&=\frac{P(k\ |\ C)\cdot P(C\ |\ k-1)}{ P(C\ |\ k-1) \cdot P(k\ |\ C) +  P(B\ |\ k-1) \cdot P(k\ |\ B)}
\end{align*}

The formula above can be used to iteratively update $P(C,k)$, and hence $P(B,k)=1-P(C,k)$, but first we must discuss $P(k\ |\ B)$.  It is important to note that when this quantity is used to update $P(C\ |\ k)$, it is already known with certainty that $f(x_0) = \cdots = f(x_{k-2})$, \textit{i.e.} $P(k-1) = 1$.  $P(k\ |\ B)$ is the probability that, given this information, evaluating $f$ one more time, at $x_{k-1}$, yields another value in agreement with $f(x_0),\cdots,f(x_{k-2})$.   We evaluate this by separating the two possible outcomes of evaluation and counting the number of balanced functions satisfying the hypotheses that would produce them.   If $f(x_{k-1}) =  f(x_0)$, then $x_{k-1}$ is the $k$-th value on which $f$ agrees.  There are $\binom{n-k}{n/2-k}$ balanced functions which would produce this result, corresponding to the selections of $n/2-k$ more of the remaining $n-k$ values on which $f$ can agree.  If $f(x_{k-1})\neq f(x_0)$, then $f$ must still agree on $n/2-k+1$ of the remaining $n-k$ values.  There are $\binom{n-k}{n/2-k+1}$ balanced functions that would produce this result. So:
\begin{align*}
P(k,B)&=\frac{\binom{n-k}{n/2-k}}{\binom{n-k}{n/2-k}+\binom{n-k}{n/2-k+1}} = \frac{\binom{n-k}{n/2-k}}{\binom{n-k+1}{n/2-k+1}} =  \frac{n/2-k+1}{n-k+1} = \frac{n-2k+2}{2n-2k+2}
\end{align*}

We are finally in a position to calculate $P(C\ |\ k)$.  Unfortunately, for fixed $n$, the machinery above does not produce formulas of bounded complexity as $k$ grows.  Each formula will be a rational function with equal degree in numerator and denominator, but those degrees seem to be $\lfloor k/2\rfloor$.  The coefficients of the leading terms show some structure that can be used for asymptotic analysis, which we do below.  We illustrate the calculation of $P(C\ |\ 2)$, $P(C\ |\ 3)$, and $P(C\ |\ 4)$, and list formulas for $P(C\ |\ 5)$ through $P(C\ |\ 7)$ ,  then discuss the results and some experimental confirmation.

\begin{align*}
P(C\ |\  2)&=\frac{P(2\ |\ C)\cdot P(C\ |\ 1)}{ P(C\ |\ 1) \cdot P(2\ |\ C) +  P(B\ |\ 1) \cdot P(2\ |\ B)} \\
&=\frac{1\cdot\frac12}{\frac12\cdot1+\frac12\cdot\frac{n-2}{2n-2}} \\
&=\frac{1}{1+\frac{n-2}{2n-2}} \\
&=\frac{2n-2}{3n-4} \\
P(C\ |\ 3)&=\frac{P(3\ |\ C)\cdot P(C\ |\ 2)}{ P(C\ |\ 2) \cdot P(3\ |\ C) +  P(B\ |\ 2) \cdot P(3\ |\ B)} \\
&=\frac{1\cdot\frac{2n-2}{3n-4}}{\frac{2n-2}{3n-4}+\left(1-\frac{2n-2}{3n-4}\right)\cdot\frac{n-4}{2n-4}} \\
&=\frac{4n-4}{5n-8} \\
P(C\ |\ 4)&=\frac{P(4\ |\ C)\cdot P(C\ |\ 3)}{ P(C\ |\ 3) \cdot P(4\ |\ C) +  P(B\ |\ 3) \cdot P(4\ |\ B)} \\
&=\frac{1\cdot\frac{4n-4}{5n-8}}{\frac{4n-4}{5n-8}+\left(1-\frac{4n-4}{5n-8}\right)\cdot\frac{n-6}{2n-6}} \\
&=\frac{8n^2-32n+24}{9n^2-42n+48} \\
P(C\ |\ 5)&=\frac{16n^2-64n+48}{17n^2-78n+96} \\
P(C\ |\ 6)&=\frac{32n^3-288n^2+736n-480}{33n^3-312n^2+924n-960} \\
P(C\ |\ 7)&=\frac{64n^3-576n^2+1472n-960}{65n^3-606n^2+1768n-1920}
\end{align*}

There are clearly patterns, the most striking of which yields $P(C\ |\ k)\xrightarrow[n\shortrightarrow\infty]{}\frac{2^{k-1}}{2^{k-1}+1}$, that is, given $k\geq2$ evaluations in agreement, the probability that $f$ is constant is ${\sim}1-\frac{1}{2^{k-1}+1}$, at least for large $n$.  In the quantum context, where $n$ is likely to be exponential in the number of qubits, this asymptotic value would be approached rapidly.  To confirm this analysis, a python script is included in the repo which experimentally calculates empirical values of $P(C\ |\ k)$ for specified values of $n$ and $k$.  It also calculates the theoretical values, recursing over $k$, for comparison.  See \href{https://github.com/tlesaul2/SolutionQCQINielsenChuang/blob/master/Python/Problem1.1.py}{\texttt{<git repo>/Python/Problem1.1.py}}.

To answer the problem most directly, \textit{i.e.}, ``what is the performance of the best classical algorithm for this problem?'', let $n$ be fixed and $0<\epsilon<\frac{1}{2}$ be specified.  The ``performance'' of classically evaluating the function of $n$ inputs in order to declare it constant with error less than $\epsilon$  is equivalent to determining the number $k$ of evaluations in agreement after which the probability that $f$ is constant is greater than $1-\epsilon$.  Note that in no case is this number less than two.  The entries in the table below are such values, with rows indexed by $n$, and columns corresponding to exponentially decreasing values of $\epsilon$.  Specifically, column $i$ lists the values of $k$ corresponding to $\epsilon = 1/2^i$.  The maximum value of $k$  in each row is $n/2+1$, since this implies the function is constant.

% Note, the indexing in the \cline's is for the
\begin{adjustbox}{center}
\begin{tabular}{|c||c|c|c|c|c|c|c|c|c|c|c|c|c|c|c|c|c|c|c|c|}
\hline
$\epsilon=1/2^i; i=$ & 1 & 2 & 3 & 4 & 5 & 6 & 7 & 8 & 9 & 10 & 11 & 12 & 13 & 14 & 15 & 16 & 17 & 18 & 19 & 20  \\ \hline \hline
$n=6$ & 2 & 3 & 3 & 4 & $\shortrightarrow$ \\ \cline{1-8}
$n=8$ & 2 & 3 & 4 & 4 & 4 & 5 & $\shortrightarrow$ \\ \cline{1-9}
$n=10$ & 2 & 3 & 4 & 4 & 5 & 5 & 6 & $\shortrightarrow$ \\ \cline{1-11}
$n=12$ & 2 & 3 & 4 & 4 & 5 & 5 & 6 & 6 & 7 & $\shortrightarrow$\\ \cline{1-13}
$n=14$ & 2 & 3 & 4 & 5 & 5 & 6 & 6 & 7 & 7 & 7 & 8 & $\shortrightarrow$ \\ \cline{1-15}
$n=16$ & 2 & 3 & 4 & 5 & 5 & 6 & 6 & 7 & 7 & 8 & 8 & 8 & 9 & $\shortrightarrow$ \\ \cline{1-17}
$n=18$ & 2 & 3 & 4 & 5 & 5 & 6 & 7 & 7 & 8 & 8 & 8 & 9 & 9 & 9 & 10 & $\shortrightarrow$ \\ \cline{1-19}
$n=20$ & 2 & 3 & 4 & 5 & 6 & 6 & 7 & 7 & 8 & 8 & 9 & 9 & 9 & 10 & 10 & 10 & 11 & $\shortrightarrow$ \\ \hline
$n=22$ & 2 & 3 & 4 & 5 & 6 & 6 & 7 & 7 & 8 & 9 & 9 & 9 & 10 & 10 & 11 & 11 & 11 & 11 & 12 & $\shortrightarrow$ \\ \hline
$n=24$ & 2 & 3 & 4 & 5 & 6 & 6 & 7 & 8 & 8 & 9 & 9 & 10 & 10 & 11 & 11 & 11 & 12 & 12 & 12 & 12 \\ \hline
$n=26$ & 2 & 3 & 4 & 5 & 6 & 6 & 7 & 8 & 8 & 9 & 9 & 10 & 10 & 11 & 11 & 12 & 12 & 12 & 13 & 13 \\ \hline
$n=28$ & 2 & 3 & 4 & 5 & 6 & 7 & 7 & 8 & 8 & 9 & 10 & 10 & 11 & 11 & 12 & 12 & 12 & 13 & 13 & 13 \\ \hline
$n=30$ & 2 & 3 & 4 & 5 & 6 & 7 & 7 & 8 & 9 & 9 & 10 & 10 & 11 & 11 & 12 & 12 & 13 & 13 & 13 & 14 \\ \hline
$n=32$ & 2 & 3 & 4 & 5 & 6 & 7 & 7 & 8 & 9 & 9 & 10 & 11 & 11 & 12 & 12 & 13 & 13 & 13 & 14 & 14 \\ \hline
$n=34$ & 2 & 3 & 4 & 5 & 6 & 7 & 7 & 8 & 9 & 9 & 10 & 11 & 11 & 12 & 12 & 13 & 13 & 14 & 14  & 15 \\ \hline
$n=36$ & 2 & 3 & 4 & 5 & 6 & 7 & 7 & 8 & 9 & 10 & 10 & 11 & 11 & 12 & 12 & 13 & 13 & 14 & 14 & 15 \\ \hline
$n=38$ & 2 & 3 & 4 & 5 & 6 & 7 & 8 & 8 & 9 & 10 & 10 & 11 & 12 & 12 & 13 & 13 & 14 & 14 & 15 & 15 \\ \hline
$n=40$ & 2 & 3 & 4 & 5 & 6 & 7 & 8 & 8 & 9 & 10 & 10 & 11 & 12 & 12 & 13 & 13 & 14 & 14 & 15 & 15 \\ \hline
$n=42$ & 2 & 3 & 4 & 5 & 6 & 7 & 8 & 8 & 9 & 10 & 10 & 11 & 12 & 12 & 13 & 14 & 14 & 15 & 15 & 16 \\ \hline
$n=44$ & 2 & 3 & 4 & 5 & 6 & 7 & 8 & 8 & 9 & 10 & 11 & 11 & 12 & 13 & 13 & 14 & 14 & 15 & 15 & 16 \\ \hline
$n=46$ & 2 & 3 & 4 & 5 & 6 & 7 & 8 & 8 & 9 & 10 & 11 & 11 & 12 & 13 & 13 & 14 & 14 & 15 & 16 & 16 \\ \hline
$n=48$ & 2 & 3 & 4 & 5 & 6 & 7 & 8 & 8 & 9 & 10 & 11 & 11 & 12 & 13 & 13 & 14 & 15 & 15 & 16 & 16 \\ \hline
$n=50$ & 2 & 3 & 4 & 5 & 6 & 7 & 8 & 9 & 9 & 10 & 11 & 11 & 12 & 13 & 13 & 14 & 15 & 15 & 16 & 16 \\ \hline
$n=52$ & 2 & 3 & 4 & 5 & 6 & 7 & 8 & 9 & 9 & 10 & 11 & 12 & 12 & 13 & 14 & 14 & 15 & 15 & 16 & 17 \\ \hline
$n=54$ & 2 & 3 & 4 & 5 & 6 & 7 & 8 & 9 & 9 & 10 & 11 & 12 & 12 & 13 & 14 & 14 & 15 & 16 & 16 & 17 \\ \hline
$n=56$ & 2 & 3 & 4 & 5 & 6 & 7 & 8 & 9 & 9 & 10 & 11 & 12 & 12 & 13 & 14 & 14 & 15 & 16 & 16 & 17 \\ \hline
$n=58$ & 2 & 3 & 4 & 5 & 6 & 7 & 8 & 9 & 9 & 10 & 11 & 12 & 12 & 13 & 14 & 15 & 15 & 16 & 16 & 17 \\ \hline
$n=60$ & 2 & 3 & 4 & 5 & 6 & 7 & 8 & 9 & 9 & 10 & 11 & 12 & 13 & 13 & 14 & 15 & 15 & 16 & 17 & 17 \\ \hline
$n=62$ & 2 & 3 & 4 & 5 & 6 & 7 & 8 & 9 & 10 & 10 & 11 & 12 & 13 & 13 & 14 & 15 & 15 & 16 & 17 & 17 \\ \hline
$n=64$ & 2 & 3 & 4 & 5 & 6 & 7 & 8 & 9 & 10 & 10 & 11 & 12 & 13 & 13 & 14 & 15 & 15 & 16 & 17 & 17 \\ \hline
\end{tabular}
\end{adjustbox}
\\

Loosely, for small numbers of evaluations and large $n$, \textit{i.e} in the bottom left of the table, each exponential increase in the probability of being constant desired requires an additional evaluation.  Eventually, the combinatorial reduction in the number of remaining balanced functions allows additonal evaluations to reduce the error with which the function can be declared constant by several powers of 2, as often seen in the top right.  That is not to say that the probability is always reduced by at least a factor of 2.  In fact, note that $P(C\ |\ 1) = 1/2$, and $P(C\ |\ 2)\xrightarrow[n\shortrightarrow\infty]{}2/3$, so $\frac{1-P(C\ |\ 1)}{1-P(C\ |\ 2)}\xrightarrow[n\shortrightarrow\infty]{}3/2$. The second evaluation only reduces the probability that the function is balanced by a factor of ${\sim}1.5$ for large $n$.  Asymptotically, for $k\geq2$, note that $\frac{1-P(C\ |\ k+1)}{1-P(C\ |\ k)}\xrightarrow[n\shortrightarrow\infty]{}\frac{\frac{1}{2^k+1}}{\frac{1}{2^{k-1}+1}}=\frac{2^{k-1}+1}{2^k+1}<2$, so all $k$-th evaluations eventually reduce the probability of the function being balanced by less than a factor of $2$, for large enough $n$.  Theoretically, it is seemingly possible there's a case in which halving the probability of being balanced requires two additional evaluations.  That is, there could exist $n$ and an $\epsilon=\frac{1}{2^i}$ requiring $k$ measurements to declare the function constant with error less than $\epsilon$ and at least $k+2$ measurements to declare the function constant with error less than $\epsilon/2$.  However, attempts to search for such a pathological case have come up empty.  The asymptotic short-fallings are overcome by the combinatorial reduction fast enough, before a power of 1/2 straddles two values of $k$. It is likely that more careful analysis could refute the possibility rigorously.

We finish discussion of this problem with a (perhaps unnecessary) table of values of $P(C\ |\ k)$ for fixed $n$ and $k$ (programmatically constructed with the python script mentioned above, as was the previous table.)  Again, once $k=n/2+1$, the function must be constant, so all probabilities are 1.

\begin{landscape}
\begin{table}

\setlength\tabcolsep{2pt}
\renewcommand{\arraystretch}{1.125}
\begin{tabular}{|c||c|c|c|c|c|c|c|c|c|c}
\hline
$k$ & 2 & 3 & 4 & 5 & 6 & 7 & 8 & 9 & 10 \\ \hline \hline
$n=4$ & $\frac{3}{4}\simeq0.7500$ & $\frac{1}{1}\simeq1.0000$ & $\rightarrow$ \\ \cline{1-5}
$n=6$ & $\frac{5}{7}\simeq0.7143$ & $\frac{10}{11}\simeq0.9091$ & $\frac{1}{1}\simeq1.0000$ & $\rightarrow$ \\ \cline{1-6}
$n=8$ & $\frac{7}{10}\simeq0.7000$ & $\frac{7}{8}\simeq0.8750$ & $\frac{35}{36}\simeq0.9722$ & $\frac{1}{1}\simeq1.0000$ & $\rightarrow$ \\ \cline{1-7}
$n=10$ & $\frac{9}{13}\simeq0.6923$ & $\frac{6}{7}\simeq0.8571$ & $\frac{21}{22}\simeq0.9545$ & $\frac{126}{127}\simeq0.9921$ & $\frac{1}{1}\simeq1.0000$ & $\rightarrow$ \\ \cline{1-8}
$n=12$ & $\frac{11}{16}\simeq0.6875$ & $\frac{11}{13}\simeq0.8462$ & $\frac{33}{35}\simeq0.9429$ & $\frac{66}{67}\simeq0.9851$ & $\frac{462}{463}\simeq0.9978$ & $\frac{1}{1}\simeq1.0000$ & $\rightarrow$ \\ \cline{1-9}
$n=14$ & $\frac{13}{19}\simeq0.6842$ & $\frac{26}{31}\simeq0.8387$ & $\frac{143}{153}\simeq0.9346$ & $\frac{143}{146}\simeq0.9795$ & $\frac{429}{431}\simeq0.9954$ & $\frac{1716}{1717}\simeq0.9994$ & $\frac{1}{1}\simeq1.0000$ & $\rightarrow$ \\ \hline
$n=16$ & $\frac{15}{22}\simeq0.6818$ & $\frac{5}{6}\simeq0.8333$ & $\frac{13}{14}\simeq0.9286$ & $\frac{39}{40}\simeq0.9750$ & $\frac{143}{144}\simeq0.9931$ & $\frac{715}{716}\simeq0.9986$ & $\frac{6435}{6436}\simeq0.9998$ & $\frac{1}{1}\simeq1.0000$ & $\rightarrow$ \\ \hline
$n=18$ & $\frac{17}{25}\simeq0.6800$ & $\frac{34}{41}\simeq0.8293$ & $\frac{85}{92}\simeq0.9239$ & $\frac{34}{35}\simeq0.9714$ & $\frac{221}{223}\simeq0.9910$ & $\frac{442}{443}\simeq0.9977$ & $\frac{2431}{2432}\simeq0.9996$ & $\frac{24310}{24311}\simeq1.0000$ & $\frac{1}{1}\simeq1.0000$ \\ \hline
$n=20$ & $\frac{19}{28}\simeq0.6786$ & $\frac{19}{23}\simeq0.8261$ & $\frac{323}{351}\simeq0.9202$ & $\frac{646}{667}\simeq0.9685$ & $\frac{646}{653}\simeq0.9893$ & $\frac{323}{324}\simeq0.9969$ & $\frac{4199}{4202}\simeq0.9993$ & $\frac{8398}{8399}\simeq0.9999$ & $\frac{92378}{92379}\simeq1.0000$ \\ \hline
$n=22$ & $\frac{21}{31}\simeq0.6774$ & $\frac{14}{17}\simeq0.8235$ & $\frac{133}{145}\simeq0.9172$ & $\frac{57}{59}\simeq0.9661$ & $\frac{323}{327}\simeq0.9878$ & $\frac{1292}{1297}\simeq0.9961$ & $\frac{969}{970}\simeq0.9990$ & $\frac{4522}{4523}\simeq0.9998$ & $\frac{29393}{29394}\simeq1.0000$ \\ \hline
$n=24$ & $\frac{23}{34}\simeq0.6765$ & $\frac{23}{28}\simeq0.8214$ & $\frac{161}{176}\simeq0.9148$ & $\frac{161}{167}\simeq0.9641$ & $\frac{437}{443}\simeq0.9865$ & $\frac{437}{439}\simeq0.9954$ & $\frac{7429}{7439}\simeq0.9987$ & $\frac{14858}{14863}\simeq0.9997$ & $\frac{14858}{14859}\simeq0.9999$ \\ \hline
$n=26$ & $\frac{25}{37}\simeq0.6757$ & $\frac{50}{61}\simeq0.8197$ & $\frac{115}{126}\simeq0.9127$ & $\frac{230}{239}\simeq0.9623$ & $\frac{805}{817}\simeq0.9853$ & $\frac{575}{578}\simeq0.9948$ & $\frac{10925}{10943}\simeq0.9984$ & $\frac{2185}{2186}\simeq0.9995$ & $\frac{37145}{37149}\simeq0.9999$ \\ \hline
$n=28$ & $\frac{27}{40}\simeq0.6750$ & $\frac{9}{11}\simeq0.8182$ & $\frac{225}{247}\simeq0.9109$ & $\frac{270}{281}\simeq0.9609$ & $\frac{690}{701}\simeq0.9843$ & $\frac{345}{347}\simeq0.9942$ & $\frac{1035}{1037}\simeq0.9981$ & $\frac{1725}{1726}\simeq0.9994$ & $\frac{6555}{6556}\simeq0.9998$ \\ \hline
$n=30$ & $\frac{29}{43}\simeq0.6744$ & $\frac{58}{71}\simeq0.8169$ & $\frac{261}{287}\simeq0.9094$ & $\frac{261}{272}\simeq0.9596$ & $\frac{1305}{1327}\simeq0.9834$ & $\frac{1740}{1751}\simeq0.9937$ & $\frac{10005}{10027}\simeq0.9978$ & $\frac{10005}{10012}\simeq0.9993$ & $\frac{10005}{10007}\simeq0.9998$ \\ \hline
$n=32$ & $\frac{31}{46}\simeq0.6739$ & $\frac{31}{38}\simeq0.8158$ & $\frac{899}{990}\simeq0.9081$ & $\frac{899}{938}\simeq0.9584$ & $\frac{8091}{8234}\simeq0.9826$ & $\frac{8091}{8146}\simeq0.9932$ & $\frac{4495}{4506}\simeq0.9976$ & $\frac{13485}{13496}\simeq0.9992$ & $\frac{310155}{310232}\simeq0.9998$ \\ \hline
$n=34$ & $\frac{33}{49}\simeq0.6735$ & $\frac{22}{27}\simeq0.8148$ & $\frac{341}{376}\simeq0.9069$ & $\frac{2046}{2137}\simeq0.9574$ & $\frac{9889}{10071}\simeq0.9819$ & $\frac{1798}{1811}\simeq0.9928$ & $\frac{24273}{24338}\simeq0.9973$ & $\frac{5394}{5399}\simeq0.9991$ & $\frac{13485}{13489}\simeq0.9997$ \\ \hline
$n=36$ & $\frac{35}{52}\simeq0.6731$ & $\frac{35}{43}\simeq0.8140$ & $\frac{77}{85}\simeq0.9059$ & $\frac{22}{23}\simeq0.9565$ & $\frac{682}{695}\simeq0.9813$ & $\frac{1705}{1718}\simeq0.9924$ & $\frac{4495}{4508}\simeq0.9971$ & $\frac{12586}{12599}\simeq0.9990$ & $\frac{37758}{37771}\simeq0.9997$ \\ \hline
$n=38$ & $\frac{37}{55}\simeq0.6727$ & $\frac{74}{91}\simeq0.8132$ & $\frac{1295}{1431}\simeq0.9050$ & $\frac{259}{271}\simeq0.9557$ & $\frac{407}{415}\simeq0.9807$ & $\frac{1628}{1641}\simeq0.9921$ & $\frac{12617}{12656}\simeq0.9969$ & $\frac{11470}{11483}\simeq0.9989$ & $\frac{33263}{33276}\simeq0.9996$ \\ \hline
$n=40$ & $\frac{39}{58}\simeq0.6724$ & $\frac{13}{16}\simeq0.8125$ & $\frac{481}{532}\simeq0.9041$ & $\frac{1443}{1511}\simeq0.9550$ & $\frac{3367}{3435}\simeq0.9802$ & $\frac{481}{485}\simeq0.9918$ & $\frac{1221}{1225}\simeq0.9967$ & $\frac{814}{815}\simeq0.9988$ & $\frac{2294}{2295}\simeq0.9996$ \\ \hline
$n=42$ & $\frac{41}{61}\simeq0.6721$ & $\frac{82}{101}\simeq0.8119$ & $\frac{533}{590}\simeq0.9034$ & $\frac{1066}{1117}\simeq0.9543$ & $\frac{19721}{20129}\simeq0.9797$ & $\frac{19721}{19891}\simeq0.9915$ & $\frac{19721}{19789}\simeq0.9966$ & $\frac{1517}{1519}\simeq0.9987$ & $\frac{16687}{16695}\simeq0.9995$ \\ \hline
$n=44$ & $\frac{43}{64}\simeq0.6719$ & $\frac{43}{53}\simeq0.8113$ & $\frac{1763}{1953}\simeq0.9027$ & $\frac{3526}{3697}\simeq0.9537$ & $\frac{45838}{46807}\simeq0.9793$ & $\frac{22919}{23123}\simeq0.9912$ & $\frac{848003}{851063}\simeq0.9964$ & $\frac{848003}{849193}\simeq0.9986$ & $\frac{65231}{65265}\simeq0.9995$ \\ \hline
$n=46$ & $\frac{45}{67}\simeq0.6716$ & $\frac{30}{37}\simeq0.8108$ & $\frac{129}{143}\simeq0.9021$ & $\frac{387}{406}\simeq0.9532$ & $\frac{1763}{1801}\simeq0.9789$ & $\frac{35260}{35583}\simeq0.9909$ & $\frac{343785}{345077}\simeq0.9963$ & $\frac{22919}{22953}\simeq0.9985$ & $\frac{848003}{848479}\simeq0.9994$ \\ \hline
$n=48$ & $\frac{47}{70}\simeq0.6714$ & $\frac{47}{58}\simeq0.8103$ & $\frac{705}{782}\simeq0.9015$ & $\frac{141}{148}\simeq0.9527$ & $\frac{6063}{6196}\simeq0.9785$ & $\frac{2021}{2040}\simeq0.9907$ & $\frac{82861}{83184}\simeq0.9961$ & $\frac{414305}{414951}\simeq0.9984$ & $\frac{1077193}{1077839}\simeq0.9994$ \\ \hline
$n=50$ & $\frac{49}{73}\simeq0.6712$ & $\frac{98}{121}\simeq0.8099$ & $\frac{2303}{2556}\simeq0.9010$ & $\frac{658}{691}\simeq0.9522$ & $\frac{987}{1009}\simeq0.9782$ & $\frac{1974}{1993}\simeq0.9905$ & $\frac{14147}{14204}\simeq0.9960$ & $\frac{198058}{198381}\simeq0.9984$ & $\frac{4060189}{4062773}\simeq0.9994$ \\ \hline
$n=52$ & $\frac{51}{76}\simeq0.6711$ & $\frac{17}{21}\simeq0.8095$ & $\frac{833}{925}\simeq0.9005$ & $\frac{4998}{5251}\simeq0.9518$ & $\frac{11186}{11439}\simeq0.9779$ & $\frac{5593}{5648}\simeq0.9903$ & $\frac{50337}{50546}\simeq0.9959$ & $\frac{11186}{11205}\simeq0.9983$ & $\frac{28294}{28313}\simeq0.9993$ \\ \hline
$n=54$ & $\frac{53}{79}\simeq0.6709$ & $\frac{106}{131}\simeq0.8092$ & $\frac{901}{1001}\simeq0.9001$ & $\frac{901}{947}\simeq0.9514$ & $\frac{44149}{45161}\simeq0.9776$ & $\frac{25228}{25481}\simeq0.9901$ & $\frac{296429}{297694}\simeq0.9958$ & $\frac{592858}{593903}\simeq0.9982$ & $\frac{296429}{296638}\simeq0.9993$ \\ \hline
$n=56$ & $\frac{55}{82}\simeq0.6707$ & $\frac{55}{68}\simeq0.8088$ & $\frac{583}{648}\simeq0.8997$ & $\frac{583}{613}\simeq0.9511$ & $\frac{9911}{10141}\simeq0.9773$ & $\frac{4505}{4551}\simeq0.9899$ & $\frac{31535}{31673}\simeq0.9956$ & $\frac{12614}{12637}\simeq0.9982$ & $\frac{592858}{593295}\simeq0.9993$ \\ \hline
$n=58$ & $\frac{57}{85}\simeq0.6706$ & $\frac{38}{47}\simeq0.8085$ & $\frac{1045}{1162}\simeq0.8993$ & $\frac{1254}{1319}\simeq0.9507$ & $\frac{11077}{11337}\simeq0.9771$ & $\frac{11077}{11192}\simeq0.9897$ & $\frac{51357}{51587}\simeq0.9955$ & $\frac{85595}{85756}\simeq0.9981$ & $\frac{119833}{119925}\simeq0.9992$ \\ \hline
$n=60$ & $\frac{59}{88}\simeq0.6705$ & $\frac{59}{73}\simeq0.8082$ & $\frac{1121}{1247}\simeq0.8990$ & $\frac{2242}{2359}\simeq0.9504$ & $\frac{24662}{25247}\simeq0.9768$ & $\frac{12331}{12461}\simeq0.9896$ & $\frac{653543}{656533}\simeq0.9954$ & $\frac{59413}{59528}\simeq0.9981$ & $\frac{1010021}{1010826}\simeq0.9992$ \\ \hline
$n=62$ & $\frac{61}{91}\simeq0.6703$ & $\frac{122}{151}\simeq0.8079$ & $\frac{3599}{4005}\simeq0.8986$ & $\frac{3599}{3788}\simeq0.9501$ & $\frac{68381}{70019}\simeq0.9766$ & $\frac{273524}{276449}\simeq0.9894$ & $\frac{752191}{755701}\simeq0.9954$ & $\frac{752191}{753686}\simeq0.9980$ & $\frac{3624193}{3627183}\simeq0.9992$ \\ \hline
$n=64$ & $\frac{63}{94}\simeq0.6702$ & $\frac{21}{26}\simeq0.8077$ & $\frac{1281}{1426}\simeq0.8983$ & $\frac{549}{578}\simeq0.9498$ & $\frac{3599}{3686}\simeq0.9764$ & $\frac{3599}{3638}\simeq0.9893$ & $\frac{68381}{68706}\simeq0.9953$ & $\frac{478667}{479642}\simeq0.9980$ & $\frac{5265337}{5269822}\simeq0.9991$ \\ \hline

\end{tabular}
\end{table}
\end{landscape}


